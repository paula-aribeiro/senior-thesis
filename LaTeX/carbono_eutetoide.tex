\documentclass{article}
\usepackage[a4paper,margin=2.5cm]{geometry}

\usepackage[utf8]{inputenc}
\usepackage[T1]{fontenc}
\usepackage[symbolgreek]{mathastext}
\usepackage{amsmath}
\usepackage{hyperref}
\usepackage{textcomp}
\usepackage{times}

\begin{document}

Modelo de segunda ordem:

\begin{equation}
	y = \beta_0 + \sum_{i=1}^k \beta_i x_i + \sum_{i=1}^{k} \sum_{j=i}^k \beta_{ij} x_i x_j + \epsilon
\end{equation}

% x1*x1 x1*x2 x1*x3 x1*x4 x1*x5
% x2*x2 x2*x3 x2*x4 x2*x5
% x3*x3 x3*x4 x3*x5
% x4*x4 x4*x5
% x5*x5

% Alternativamente:

% \begin{equation}
% 	y = \beta_0 + \sum_{i=1}^k \beta_i x_i + \sum_{i=1}^k \beta_{ii} x_i^2 + \sum_{i=1}^{k-1} \sum_{j=i+1}^k \beta_{ij} x_i x_j + \epsilon
% \end{equation}

% % x1*x2 x1*x3 x1*x4 x1*x5
% % x2*x3 x2*x4 x2*x5
% % x3*x4 x3*x5
% % x4*x5

Equações para $A_3$:

\begin{align}
	A_3^{hipo} &= \beta_0^{hipo} + \sum_{i=1}^k \beta_i^{hipo} x_i + \sum_{i=1}^{k} \sum_{j=i}^k \beta_{ij}^{hipo} x_i x_j + \epsilon \\
	A_3^{hiper} &= \beta_0^{hiper} + \sum_{i=1}^k \beta_i^{hiper} x_i + \sum_{i=1}^{k} \sum_{j=i}^k \beta_{ij}^{hipo} x_i x_j + \epsilon
\end{align}

No ponto eutetóide:

\begin{equation}
	A_3^{hipo} = A_3^{hiper} \Rightarrow A_3^{hipo} - A_3^{hiper} = 0
\end{equation}

Logo:

\begin{equation}
	\beta_0^{hipo} - \beta_0^{hiper} 
	+ \sum_{i=1}^k \left(\beta_i^{hipo} - \beta_i^{hiper}\right) x_i 
	+ \sum_{i=1}^{k} \sum_{j=i}^k \left(\beta_{ij}^{hipo} - \beta_{ij}^{hiper}\right) x_i x_j = 0
	\label{eq:eutetoide_1}
\end{equation}

Definamos:

\begin{align}
	\beta_i^{eut} &= \beta_i^{hipo} - \beta_i^{hiper} \\
	\beta_{ij}^{eut} &= \beta_{ij}^{hipo} - \beta_{ij}^{hiper}
\end{align}

Substituindo as definições acima na equação \ref{eq:eutetoide_1}, temos:

\begin{equation}
	\beta_0^{eut} + \sum_{i=1}^k \beta_i^{eut} x_i 
	+ \sum_{i=1}^{k} \sum_{j=i}^k \beta_{ij}^{eut} x_i x_j = 0
	\label{eq:eutetoide_2}
\end{equation}

Reescrevendo \ref{eq:eutetoide_2} de modo a separar $x_1$ (fração de carbono):

\begin{align}
	&\beta_0^{eut} 
	+ \beta_1^{eut} x_ 1 
	+ \sum_{i=2}^k \beta_i^{eut} x_i 
	+ \beta_{11}^{eut} {x_1}^2 
	+ \sum_{j=2}^k \beta_{1j}^{eut} x_1 x_j
	+ \sum_{i=2}^{k} \sum_{j=i}^k \beta_{ij}^{eut} x_i x_j = 0 \nonumber \\
	&\beta_{11}^{eut} {x_1}^2 
	+ \left(\beta_1^{eut} + \sum_{j=2}^k \beta_{1j}^{eut} x_j \right) x_1 
	+ \left(\beta_0^{eut} 
	+ \sum_{i=2}^k \beta_i^{eut} x_i 
	+ \sum_{i=2}^{k} \sum_{j=i}^k \beta_{ij}^{eut} x_i x_j\right) = 0 \label{eq:eutetoide_3}
\end{align}

Dividindo todos os termos em \ref{eq:eutetoide_3} por $2 \beta_{11}^{eut}$ e definindo:

\begin{align}
	\alpha_i 
	&= \frac{\beta_i^{hipo} - \beta_i^{hiper}}{2 \beta_{11}}
	= \frac{1}{2} \frac{\beta_i^{hipo} - \beta_i^{hiper}}{\beta_{11}^{hipo} - \beta_{11}^{hiper}}\\
	\alpha_{ij} 
	&= \frac{\beta_{ij}^{hipo} - \beta_{ij}^{hiper}}{2 \beta_{11}}
	= \frac{1}{2} \frac{\beta_{ij}^{hipo} - \beta_{ij}^{hiper}}{\beta_{11}^{hipo} - \beta_{11}^{hiper}}
\end{align}
%
então obtemos:

\begin{equation}
	\frac{{x_1}^2}{2}
	+ \left(\alpha_1 + \sum_{j=2}^k \alpha_{1j} x_j \right) x_1 
	+ \left(\alpha_0 
	+ \sum_{i=2}^k \alpha_i x_i 
	+ \sum_{i=2}^{k} \sum_{j=i}^k \alpha_{ij} x_i x_j\right) = 0 \label{eq:eutetoide_4}
\end{equation}

A equação \ref{eq:eutetoide_4} pode ser resolvida para $x_1$ utilizando a fórmula quadrática:

\begin{equation}
	x_1 = \frac{-b \pm \sqrt{b^2 - 4 a c}}{2 a}
	\label{eq:quadratica}
\end{equation}
%
em que $a = 1/2$ e $b$ e $c$ são:

\begin{align}
	b &= \alpha_1 + \sum_{j=2}^k \alpha_{1j} x_j \\
	c &= \alpha_0 
	+ \sum_{i=2}^k \alpha_i x_i 
	+ \sum_{i=2}^{k} \sum_{j=i}^k \alpha_{ij} x_i x_j
\end{align}

Usando a substituição $a = 1/2$, a equação \ref{eq:quadratica} torna-se simplesmente:

\begin{equation}
	x_1 = -b \pm \sqrt{b^2 - 2 c}
	\label{eq:quadratica_2}
\end{equation}

% \begin{equation}
% 	-\left(\alpha_1 + \sum_{j=2}^k \alpha_{ij} x_j\right) \pm \sqrt{ \left(\alpha_1 + \sum_{j=2}^k \alpha_{ij} x_j\right)^2 - 2 \left( \alpha_0 
% 	+ \sum_{i=2}^k \alpha_i x_i 
% 	+ \sum_{i=2}^{k} \sum_{j=i}^k \alpha_{ij} x_i x_j \right)}
% \end{equation}

Em suma:

\begin{align*}
	\alpha_i &= \frac{1}{2} \frac{\beta_i^{hipo} - \beta_i^{hiper}}{\beta_{11}^{hipo} - \beta_{11}^{hiper}}\\
	\alpha_{ij} &= \frac{1}{2} \frac{\beta_{ij}^{hipo} - \beta_{ij}^{hiper}}{\beta_{11}^{hipo} - \beta_{11}^{hiper}}\\
	b &= \alpha_1 + \sum_{j=2}^k \alpha_{1j} x_j \\
	c &= \alpha_0 
	+ \sum_{i=2}^k \alpha_i x_i 
	+ \sum_{i=2}^{k} \sum_{j=i}^k \alpha_{ij} x_i x_j \\
	x_1 &= -b \pm \sqrt{b^2 - 2 c}
\end{align*}

\pagebreak

Tabela de correspondência:

\begin{table}[ht]
	\centering
	\begin{tabular}{c c c c c}
	\hline
	$x_1$ & $x_2$ & $x_3$ & $x_4$ & $x_5$ \\
	\hline
	C & Mn & Si & Cr & Ni \\
	\hline
	\end{tabular}
\end{table}

\begin{align}
	b =&\,\alpha_{C} + \alpha_{C Mn} Mn + \alpha_{C Si} Si + \alpha_{C Cr} Cr + \alpha_{C Ni} Ni\\
	c =&\,\alpha_0 + \alpha_{Mn} Mn + \alpha_{Si} Si + \alpha_{Cr} Cr + \alpha_{Ni} Ni + \nonumber \\
	   &\,\alpha_{Mn^2}\ Mn^2 + \alpha_{Mn Si}\ Mn\ Si + \alpha_{Mn Cr}\ Mn\ Cr + \alpha_{Mn Ni}\ Mn\ Ni + \nonumber \\
	   &\,\alpha_{Si^2}\ Si^2 + \alpha_{Si Cr}\ Si\ Cr + \alpha_{Si Ni}\ Si\ Ni + \nonumber \\
	   &\,\alpha_{Cr^2}\ Cr^2 + \alpha_{Cr Ni}\ Cr\ Ni + \nonumber \\
	   &\,\alpha_{Ni^2}\ Ni^2
\end{align}

\end{document}